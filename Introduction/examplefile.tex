\chapter{A Primer of Basic \LaTeX Commands for Those Just Starting Out}

An example figure looks like the one below in Figure \ref{fig:example-figure}.

\begin{figure}[ht]
    \centering
    \includegraphics[width=0.75\textwidth]{example-image-a}
    \caption{Here's an example figure}
    \label{fig:example-figure}
\end{figure}

For an example footnote, use this syntax\footnote{\href{https://www.cst.cam.ac.uk/}{The CST Department Website}}.

For bullet points:
\begin{itemize}
    \item First item.
    \item Second item.
\end{itemize}

For numerated lists:
\begin{enumerate}
    \item First item.
    \item Second item.
\end{enumerate}

See Table \ref{tab:example-table} for an example table using the booktabs package. Note the positioning of the caption above the table rather than below for figures.

\begin{table}
\centering
 \caption{List of Cambridge Colleges (sorted by most recent founding date)}
 \label{tab:example-table}
    \begin{tabular}{c c c}
        \toprule
         \textbf{\#} & \textbf{College} & \textbf{Founding Date} \\
         \midrule
            1 & Peterhouse & 1284 \\
            2 & Clare & 1326 \\
            3 & Pembroke & 1347 \\
            4 & Gonville and Caius & 1348 \\
            5 & Trinity Hall & 1350 \\
         \bottomrule
    \end{tabular}
\end{table}

To cite, we simply use the \texttt{\textbackslash citep\{\}} or \texttt{\textbackslash citet\{\}} command (see below).

Kitchen supplies are known to go missing in an office context \citep{CaseOfDisappearingTeaspoons}. However, not all hope is lost, nor should we lose faith in humankind's ability to retain teaspoons. For instance, \citet{TeaspoonSolutionProposal} has proposed an elegant solution which could be considered world-changing.

To list code, use the lstlisting package.